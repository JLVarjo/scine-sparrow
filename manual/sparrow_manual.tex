\documentclass[]{tufte-book}

\hypersetup{colorlinks}% uncomment this line if you prefer colored hyperlinks (e.g., for onscreen viewing)

%%
% For graphics / images
\usepackage{graphicx}
\usepackage{amsmath}
\setkeys{Gin}{width=\linewidth,totalheight=\textheight,keepaspectratio}
\graphicspath{{graphics/}}
\usepackage{hyperref}
\usepackage{chemformula}

%%
% Book metadata

%\newsavebox{\titleimage}
%\savebox{\titleimage}{\includegraphics[height=10\baselineskip]{graphics/autocas-icon.pdf}\hspace{1cm}\Huge Version 1.0 -- May 2018}

%\newsavebox{\scineimage}
%\savebox{\scineimage}{\includegraphics[height=5\baselineskip]{scine_darkblue.png}}


\title[SCINE Sparrow manual]{User Manual \vskip 0.5em {\setlength{\parindent}{0pt} \Huge SCINE Sparrow 3.1.0}}
\author[The SCINE Sparrow Developers]{The SCINE Sparrow Developers: \newline \noindent Francesco Bosia, Tamara Husch, Charlotte H.~M\"uller, Severin Polonius, Jan-Grimo Sobez, Miguel Steiner, Jan P.~Unsleber, Alain C.~Vaucher, Thomas Weymuth, and Markus Reiher}
\publisher{ETH Z\"urich}

%%
% If they're installed, use Bergamo and Chantilly from www.fontsite.com.
% They're clones of Bembo and Gill Sans, respectively.
%\IfFileExists{bergamo.sty}{\usepackage[osf]{bergamo}}{}% Bembo
%\IfFileExists{chantill.sty}{\usepackage{chantill}}{}% Gill Sans

%\usepackage{microtype}

%%
% Just some sample text
\usepackage{lipsum}

%%
% For nicely typeset tabular material
\usepackage{booktabs}

% The fancyvrb package lets us customize the formatting of verbatim
% environments.  We use a slightly smaller font.
\usepackage{fancyvrb}
\fvset{fontsize=\normalsize}

%%
% Prints argument within hanging parentheses (i.e., parentheses that take
% up no horizontal space).  Useful in tabular environments.
\newcommand{\hangp}[1]{\makebox[0pt][r]{(}#1\makebox[0pt][l]{)}}

%%
% Prints an asterisk that takes up no horizontal space.
% Useful in tabular environments.
\newcommand{\hangstar}{\makebox[0pt][l]{*}}

%%
% Prints a trailing space in a smart way.
\usepackage{xspace}

%%
% Some shortcuts for Tufte's book titles.  The lowercase commands will
% produce the initials of the book title in italics.  The all-caps commands
% will print out the full title of the book in italics.
\newcommand{\vdqi}{\textit{VDQI}\xspace}
\newcommand{\ei}{\textit{EI}\xspace}
\newcommand{\ve}{\textit{VE}\xspace}
\newcommand{\be}{\textit{BE}\xspace}
\newcommand{\VDQI}{\textit{The Visual Display of Quantitative Information}\xspace}
\newcommand{\EI}{\textit{Envisioning Information}\xspace}
\newcommand{\VE}{\textit{Visual Explanations}\xspace}
\newcommand{\BE}{\textit{Beautiful Evidence}\xspace}

\newcommand{\TL}{Tufte-\LaTeX\xspace}

% Prints the month name (e.g., January) and the year (e.g., 2008)
\newcommand{\monthyear}{%
  \ifcase\month\or January\or February\or March\or April\or May\or June\or
  July\or August\or September\or October\or November\or
  December\fi\space\number\year
}


% Prints an epigraph and speaker in sans serif, all-caps type.
\newcommand{\openepigraph}[2]{%
  %\sffamily\fontsize{14}{16}\selectfont
  \begin{fullwidth}
  \sffamily\large
  \begin{doublespace}
  \noindent\allcaps{#1}\\% epigraph
  \noindent\allcaps{#2}% author
  \end{doublespace}
  \end{fullwidth}
}

% Inserts a blank page
\newcommand{\blankpage}{\newpage\hbox{}\thispagestyle{empty}\newpage}

\usepackage{units}

% Typesets the font size, leading, and measure in the form of 10/12x26 pc.
\newcommand{\measure}[3]{#1/#2$\times$\unit[#3]{pc}}

% Macros for typesetting the documentation
\newcommand{\hlred}[1]{\textcolor{Maroon}{#1}}% prints in red
\newcommand{\hangleft}[1]{\makebox[0pt][r]{#1}}
\newcommand{\hairsp}{\hspace{1pt}}% hair space
\newcommand{\hquad}{\hskip0.5em\relax}% half quad space
\newcommand{\TODO}{\textcolor{red}{\bf TODO!}\xspace}
\newcommand{\ie}{\textit{i.\hairsp{}e.}\xspace}
\newcommand{\eg}{\textit{e.\hairsp{}g.}\xspace}
\newcommand{\na}{\quad--}% used in tables for N/A cells
\providecommand{\XeLaTeX}{X\lower.5ex\hbox{\kern-0.15em\reflectbox{E}}\kern-0.1em\LaTeX}
\newcommand{\tXeLaTeX}{\XeLaTeX\index{XeLaTeX@\protect\XeLaTeX}}
% \index{\texttt{\textbackslash xyz}@\hangleft{\texttt{\textbackslash}}\texttt{xyz}}
\newcommand{\tuftebs}{\symbol{'134}}% a backslash in tt type in OT1/T1
\newcommand{\doccmdnoindex}[2][]{\texttt{\tuftebs#2}}% command name -- adds backslash automatically (and doesn't add cmd to the index)
\newcommand{\doccmddef}[2][]{%
  \hlred{\texttt{\tuftebs#2}}\label{cmd:#2}%
  \ifthenelse{\isempty{#1}}%
    {% add the command to the index
      \index{#2 command@\protect\hangleft{\texttt{\tuftebs}}\texttt{#2}}% command name
    }%
    {% add the command and package to the index
      \index{#2 command@\protect\hangleft{\texttt{\tuftebs}}\texttt{#2} (\texttt{#1} package)}% command name
      \index{#1 package@\texttt{#1} package}\index{packages!#1@\texttt{#1}}% package name
    }%
}% command name -- adds backslash automatically
\newcommand{\doccmd}[2][]{%
  \texttt{\tuftebs#2}%
  \ifthenelse{\isempty{#1}}%
    {% add the command to the index
      \index{#2 command@\protect\hangleft{\texttt{\tuftebs}}\texttt{#2}}% command name
    }%
    {% add the command and package to the index
      \index{#2 command@\protect\hangleft{\texttt{\tuftebs}}\texttt{#2} (\texttt{#1} package)}% command name
      \index{#1 package@\texttt{#1} package}\index{packages!#1@\texttt{#1}}% package name
    }%
}% command name -- adds backslash automatically
\newcommand{\docopt}[1]{\ensuremath{\langle}\textrm{\textit{#1}}\ensuremath{\rangle}}% optional command argument
\newcommand{\docarg}[1]{\textrm{\textit{#1}}}% (required) command argument
\newenvironment{docspec}{\begin{quotation}\ttfamily\parskip0pt\parindent0pt\ignorespaces}{\end{quotation}}% command specification environment
\newcommand{\docenv}[1]{\texttt{#1}\index{#1 environment@\texttt{#1} environment}\index{environments!#1@\texttt{#1}}}% environment name
\newcommand{\docenvdef}[1]{\hlred{\texttt{#1}}\label{env:#1}\index{#1 environment@\texttt{#1} environment}\index{environments!#1@\texttt{#1}}}% environment name
\newcommand{\docpkg}[1]{\texttt{#1}\index{#1 package@\texttt{#1} package}\index{packages!#1@\texttt{#1}}}% package name
\newcommand{\doccls}[1]{\texttt{#1}}% document class name
\newcommand{\docclsopt}[1]{\texttt{#1}\index{#1 class option@\texttt{#1} class option}\index{class options!#1@\texttt{#1}}}% document class option name
\newcommand{\docclsoptdef}[1]{\hlred{\texttt{#1}}\label{clsopt:#1}\index{#1 class option@\texttt{#1} class option}\index{class options!#1@\texttt{#1}}}% document class option name defined
\newcommand{\docmsg}[2]{\bigskip\begin{fullwidth}\noindent\ttfamily#1\end{fullwidth}\medskip\par\noindent#2}
\newcommand{\docfilehook}[2]{\texttt{#1}\index{file hooks!#2}\index{#1@\texttt{#1}}}
\newcommand{\doccounter}[1]{\texttt{#1}\index{#1 counter@\texttt{#1} counter}}

%attempt to allow footnotes in verbatim
\usepackage{verbatim}
\newcommand{\vfchar}[1]{%
  % the usual trick for using a "variable" active character
  \begingroup\lccode`~=`#1 \lowercase{\endgroup\def~##1~}{%
    % separate the footnote mark from the footnote text
    % so the footnote mark will occupy the same space as
    % any other character
    \makebox[0.5em][l]{\footnotemark}%
    \footnotetext{##1}%
  }%
  \catcode`#1=\active
}
\newenvironment{fverbatim}[1]
 {\verbatim\vfchar{#1}}
 {\endverbatim}




% Generates the index
\usepackage{makeidx}
\makeindex

%\usepackage{natbib}
\setcitestyle{numbers,square}

\usepackage{parskip}


\begin{document}

\setlength{\parindent}{0pt}

% Front matter
\frontmatter


% r.3 full title page
\maketitle


% v.4 copyright page
\newpage
\begin{fullwidth}
~\vfill
\thispagestyle{empty}
\setlength{\parindent}{0pt}
\setlength{\parskip}{\baselineskip}
Copyright \copyright\ \the\year\ \thanklessauthor

%\par\smallcaps{Published by \thanklesspublisher}

\par\smallcaps{https://scine.ethz.ch/download/sparrow}

\par Unless required by applicable law or agreed to in writing, the software 
is distributed on an \smallcaps{``AS IS'' BASIS, WITHOUT
WARRANTIES OR CONDITIONS OF ANY KIND}, either express or implied. \index{license}

%\par\textit{First printing, \monthyear}
\end{fullwidth}

% r.5 contents
\tableofcontents

%\listoffigures

%\listoftables


%%
% Start the main matter (normal chapters)
\mainmatter

\let\cleardoublepage\clearpage
\chapter{Introduction}

The availability of fast electronic energies and gradients is essential for the SCINE project. The SCINE \textsc{Sparrow} 
module contains electronic structure models which were designed to yield electronic energies, energy gradients with 
respect to the nuclear coordinates, and Hessians rapidly. The SCINE \textsc{Sparrow} module can be driven from SCINE 
\textsc{Interactive}, SCINE \textsc{ReaDuct}, and SCINE \textsc{Chemoton}. However, as with all SCINE modules it is also 
a stand-alone program which can be applied on its own or easily interfaced to other programs.

SCINE \textsc{Sparrow} is a command-line tool that implements many popular semiempirical models. SCINE \textsc{Sparrow} 3.1.0
provides the \texttt{MNDO}, \texttt{AM1}, \texttt{RM1}, \texttt{PM3}, \texttt{PM6}, non-SCC DFTB (\texttt{DFTB0}), \texttt{DFTB2}, and \texttt{DFTB3} methods 
(open- and closed-shell formalisms are implemented). 
The application of semiempirical models usually allows for rapid calculation of electronic energies and energy gradients 
for a small molecular structure with a given charge and spin state.

In this manual, we describe the installation of the software, an example calculation as a hands-on 
introduction to the program, and the most import functions and options.\footnote{Throughout this manual, the most 
import information is displayed in the main text, whereas useful additional information is given as a side note like this one.}
A prospect on features in future releases and references for further reading are added at the end of this manual.\enlargethispage{\baselineskip}



\chapter{Obtaining the Software}
\label{ch:obtain}

\textsc{Sparrow}  is distributed as open source software in the framework of the SCINE project (\href{https://scine.ethz.ch/}{www.scine.ethz.ch}).
Visit our website (\href{https://scine.ethz.ch/download/sparrow}{www.scine.ethz.ch/download/sparrow}) to obtain the software. 

\section{System Requirements}

\textsc{Sparrow} itself has only modest requirements regarding the hardware performance. However, the underlying quantum-chemical 
calculations might become resource intensive if extremely large systems are studied.



\chapter[Installation]{Installation}
\label{ch:installation}

\textsc{Sparrow} is distributed as an open source code. In order to compile \textsc{Sparrow} from this source code, you need
\begin{itemize}
 \item A C++ compiler supporting the C++17 standard,
 \item cmake (at least version 3.9),
 \item the Boost library (at least version 1.65.0), and
 \item the Eigen3 library (at least version 3.3.2).
\end{itemize}
In order to compile the software, either directly clone the repository with git or extract the downloaded tarball, change
to the source directory and execute the following steps:
\begin{verbatim}
git submodule init
git submodule update
mkdir build install
cd build
cmake -DCMAKE_BUILD_TYPE=Release -DCMAKE_INSTALL_PREFIX=../install ..
make
make test
make install
export PATH=$PATH:<source code directory>/install/bin
\end{verbatim}
This will configure everything, compile your software, run the tests, and install the software
into the folder ``install''. Finally, it will add the \textsc{Sparrow} binary to your \texttt{PATH}, such that you can use
it without having to specify its full location. In this last command, you have to replace \texttt{<source code directory>}
with the full path where you stored the source code of \textsc{Sparrow}.

In case you need support with the setup of \textsc{Sparrow}, please contact us by writing to \href{scine@phys.chem.ethz.ch}{scine@phys.chem.ethz.ch}.



\chapter{Example Calculation}

\textsc{Sparrow} is a command-line-only binary; there is no graphical user interface. Therefore, you always work with the
\textsc{Sparrow} binary on a command line such as the Gnome Terminal or KDE Konsole.
All functionality is accessed via command line arguments. All possible command line options can be listed with the following command:
\begin{verbatim}
sparrow --help
\end{verbatim}

In order to provide a practical demonstration of the \textsc{Sparrow} program, we present here a step-by-step example calculation that guides 
you through the complete process of calculating the total electronic energy as well as the nuclear gradient and Hessian of a molecular structure 
with \textsc{Sparrow}. We start with the following
Cartesian coordinates for water:
\begin{verbatim}
3

O      -0.27939703       0.83823215       0.00973345
H      -0.52040310       1.77677325       0.21391146
H       0.54473632       0.90669722      -0.53501306
\end{verbatim}
Store these coordinates in a file named ``h2o.xyz''. Then, call \textsc{Sparrow} with the following command:
\begin{verbatim}
sparrow  --structure h2o.xyz --molecular_charge 0 --spin_multiplicity 1 --method PM6
\end{verbatim}
This will calculate the electronic energy for the neutral water molecule with PM6. You can also use the short options
\begin{verbatim}
sparrow -x h2o.xyz -c 0 -s 1 -M PM6
\end{verbatim}
to achieve the same result. If you also want to calculate the nuclear gradient, simply add the option \texttt{-{}-gradient}
or \texttt{-G}. For the Hessian, specify \texttt{-{}-hessian} or \texttt{-H}. For the atomic second derivatives (which can be used to approximate the Hessian as a block-diagonal matrix of second derivatives. Both coordinates with respect to which the energy is derived refer to the same atom), specify \texttt{-{}-atomic\_hessians} or \texttt{-A}.



\chapter{Detailed Documentation}

\section{Command Line Arguments}

In this section, the full functionality of \textsc{Sparrow} is documented, \textit{i.e.}, all possible command line arguments
are listed and explained.

\begin{itemize}
\item \texttt{-{}-structure}, \texttt{-x}: This argument specifies the structure which should be calculated. It must be given
as a path to an XYZ file.
\item \texttt{-{}-molecular\_charge}, \texttt{-c}: This is used to specify the overall charge of the system to be calculated, 
\texttt{e.g.}, \texttt{-{}-molecular\_charge 0} or \texttt{-c -1}. The default charge is zero.
Only charges between $-$20 and +20 are supported.
\item \texttt{-{}-spin\_multiplicity}, \texttt{-s}: This is used to specify the spin multiplicity of the system to be
calculated, \texttt{e.g.}, \texttt{-s 1} for a singlet state. The default multiplicity is one. The maximal spin multiplicity is 10.
\item \texttt{-{}-gradients}, \texttt{-G}: If given, the nuclear gradients will be calculated.
\item \texttt{-{}-hessian}, \texttt{-H}: If given, the Hessian and the nuclear gradients will be calculated.
\item \texttt{-{}-atomic\_hessians}, \texttt{-A}: If given, a list of the atomic second derivatives will be calculated.
\item \texttt{-{}-thermochemistry}, \texttt{-C}: If given, the heat capacities at constant volume or pressure, 
the enthalpy, the entropy, the Gibbs free enthalpy as well as the zero point vibrational energy (ZPVE) are calculated
from the total molecular partition function.
The calculation of the thermochemical properties requires the Hessian matrix: its calculation is therefore implied by setting this option.
\item \texttt{-{}-symmetry\_number}, \texttt{-S}: Specifies the point group-dependent molecular symmetry number $\sigma$ for the calculation of thermochemical properties.
It is not adapted automatically for molecules with more than two atoms. The default value is one. 
\item \texttt{-{}-temperature}, \texttt{-T}: This is used to specify the temperature in Kelvin at which the thermochemical properties are calculated.
The default temperature is 298.15\,K. The maximum allowed temperature is 10'000\,K.
\item \texttt{-{}-suppress\_normal\_modes}, \texttt{-N}: If given, the full Hessian will be printed instead of the normal 
modes and the vibrational frequencies. This option has no effect if the Hessian is not calculated (see above).
\item \texttt{-{}-bond\_orders}, \texttt{-B}: If given, the bond order matrix will be calculated.
\item \texttt{-{}-method}, \texttt{-M}: With this option, the desired calculation method can be set. Options in
\textsc{Sparrow} 3.1.0 are \texttt{MNDO}, \texttt{AM1}, \texttt{RM1}, \texttt{PM3}, \texttt{PM6}, \texttt{DFTB0}, \texttt{DFTB2}, and
\texttt{DFTB3}. By default, \texttt{PM6} is selected.
\item \texttt{-{}-output\_to\_file}, \texttt{-o}: If this option is given, the output will not only be printed to the screen, 
but also to files. By default, the energy is stored in a file named ``energy.dat'', the nuclear gradients in a file named
``gradients.dat'', the Hessian in a file named ``hessian.dat'', the atomic Hessians in a file named ``atomic\_hessians.dat'', and the
bond order matrix in a file named ``bond\_orders.dat''.
If a description is given with the option \texttt{-d} (see below), this description will also be used in the file name.
\item \texttt{-{}-description}, \texttt{-D}: This can be used to add a (short) description of the calculation. This
description will appear in the output. This allows to quickly find a certain calculation output at a later point. The
description should be enclosed by quotation marks if it is composed by more than one word, \textit{e.g.},
\texttt{-d "This is an example"}.
\item \texttt{-{}-unrestricted\_calculation}, \texttt{-u}: If this option is given, an unrestricted (UHF) calculation
will be performed. By default, a restricted calculation will be done.
\item \texttt{-{}-wavefunction}, \texttt{-W}: If this option is given, a Molden input file is generated after the
calculation to allow for the visualization of the molecular orbitals.The basis functions used for the generation of the
Molden input are defined in the files \texttt{<source code directory>/Sparrow/Sparrow/Resources/<method>/<method>-STO-6G.basis}.
All \texttt{NDDO} methods have their own STO-6G expansion which are fine tuned based on the method's parameters.
For the \texttt{DFTB} methods, the same basis set as \texttt{PM6} is used.
\item \texttt{-{}-excited\_states}, \texttt{-E}: If this option is given, an
  electronic excited states calculation will be performed. Vertical transition
    energies and transition properties result from single electron substitutions
    for DFTB0, TD-DFTB formalism for DFTB2 and DFTB3 and NDDO-CIS for all NDDO
    methods.
\item \texttt{-{}-number\_eigenstates}, \texttt{-r}: Specifies the desired number of
  excited states. Note that this option is only considered when the option \texttt{-E} (see above) is also given.
By default, one excited state is calculated.
\item \texttt{-{}-initial\_subspace\_dimension}, \texttt{-n}: Specifies the desired number of
  initial guess vectors for the calculation of the excited states with TD-DFTB
    or NDDO-CIS. Multiples of the desired number of excited states may have a
    beneficial impact on the computational time. Note that this option is only considered when the option \texttt{-E}
(see above) is also given. The default value is 0.
\item \texttt{-{}-spin\_block}, \texttt{-b}: Specifies the desired excited-state
  symmetry to compute. Possible values are \texttt{-b singlet} for singlet
    excited states, \texttt{-b triplet} for triplet excited states, and
    \texttt{-b both} if excited states of both spin symmetries are desired. This
    option has no effect if an unrestricted calculation is performed. Note that this option is only considered when the
option \texttt{-E} (see above) is also given. By default, the singlet states are computed.
\item \texttt{-{}-number\_orbital\_mixes} \texttt{-O}:
    Sets the number of orbital steers to carry out, if 0, no steering occurs.
    The default is 0, so no steering. The command
    \texttt{-{}-number\_orbital\_mixes} 3 causes the program to first do a
    reference calculation, then three rounds of random mixing of the orbitals, 
    followed by an acceptance of the lowest energy.
\item \texttt{-{}-number\_orbitals\_to\_mix}: Specifies how many
  occupied--virtual orbital pairs are
  sampled without replacement and mixed together in each mixing round of the
  orbital steering. The default is 15 orbital pairs.
\item \texttt{-{}-number\_orbitals\_to\_consider}: Specifies the number of
  orbitals around the Fermi level that will form the set from which the orbital steering 
  samples the occupied--virtual orbital pairs that can be mixed. The default is
  0, which causes the program to consider all the orbitals
\item \texttt{-{}-maximal\_mixing\_angle}: Specifies the maximal angle in
  degrees for the 
  mixing of the occupied--virtual orbital pairs in the orbital steering. The
  default value is 90 degrees, i.e. the angle for which the two orbitals are
  swapped.
\item \texttt{-{}-minimal\_mixing\_angle}: Specifies the minimal angle in
  degrees for the 
  mixing of the occupied--virtual orbital pairs in the orbital steering. The
  default value is 0 degrees, i.e. the angle for which nothing happens.
  The new, mixed orbitals are obtained according to:
  \begin{align*}
    \phi_{\mathrm {occ, new}} &= \cos \alpha \cdot \phi_{\mathrm{occ, old}} + \sin \alpha \cdot
    \phi_{\mathrm {vir, old}}\\
    \phi_{\mathrm {vir, new}} &= \cos \alpha \cdot \phi_{\mathrm{vir, old}} - \sin \alpha \cdot
    \phi_{\mathrm {occ, old}} \, .
  \end{align*}

\end{itemize}

The following options are usually not needed:

\begin{itemize}
\item \texttt{-{}-help}, \texttt{-h}: This prints a short help message listing and explaining all possible command line 
arguments
\item \texttt{-{}-scf\_mixer}, \texttt{-m}: With this option, the method used to accelerate the convergence of the self-
consistent-field (SCF) calculations can be set. Possible options are `no\_mixer` (no convergence acceleration), `diis`
(direct inversion of the iterative subspace, DIIS), `ediis` (energy DIIS), and `ediis\_diis`. The
default is `diis`.
\item \texttt{-{}-max\_scf\_iterations}, \texttt{-I}: This is used to specify the maximum number of SCF iterations. Default is 100.
\item \texttt{-{}-self\_consistence\_criterion}, \texttt{-t}: This specifies the convergence threshold for the electronic energy.
This value is given in hartree. By default, it is `1e-7`. 
\item \texttt{-{}-density\_rmsd\_criterion}: This specifies the
  convergence threshold for the density difference between iterations (density
    matrix RMSD). By default, it is `1e-5`. 
\item \texttt{-{}-method\_parameters}, \texttt{-p}: This option can be used to specify the path of the parameter file to be
used. This option is usually not needed, since \textsc{Sparrow} provides default parameter for all its methods.
\item \texttt{-{}-excited\_parameterfile}, \texttt{-j}: This option can be used
  to specify the path of the parameter file for the excited-state calculation
    with NDDO-CIS to be used. This option is usually not needed, since \textsc{Sparrow} provides default parameter for 
    the excited-state calculation with all its methods. Note that this option is only considered when the
option \texttt{-E} (see above) is also given.
\item \texttt{-{}-log}, \texttt{-l}: This sets the log level for warning messages and errors. Supported levels are 
\texttt{debug}, \texttt{output}, \texttt{warning}, \texttt{error}, and
    \texttt{none}. By default, the level is set to
\texttt{warning}, \textit{i.e.}, all warnings are printed to STDOUT and errors are printed to STDERR. If you set this option to \texttt{error}, 
only errors are printed. If you set this to \texttt{none}, neither warnings nor
    errors are printed. If you set this option to a string other than the ones
    indicated above, an exception is thrown.
\item \texttt{-{}-log\_filename}, \texttt{-f}: This sets the name of the file
  where the logging shall be printed. By default, logging to file is disabled.
\item \texttt{-{}-distance\_threshold}, \texttt{-d}: specifies the desired
  distance cutoff for the 2-electron interactions in nddo-cis excited-state
  calculations. This feature can lead to calculations not converging. Default is
    no screening.
\item \texttt{-{}-prune\_basis}, \texttt{-y}: if this option is given, the
  TD-DFTB calculation is performed in a pruned determinant basis. The pruning is 
    specified by the options \texttt{-{}-energy\_threshold} and
    \texttt{-{}-pt\_threshold}.
\item \texttt{-{}-energy\_threshold}, \texttt{-e}: specifies the desired
  energy cutoff in a.u. for the basis functions spanning the excited stated to be
    included. Every basis function characterized by a single orbital
    substitution between orbitals with an energy difference smaller than this
    threshold is included to the pruned basis.
\item \texttt{-{}-pt\_threshold}: specifies the desired
  perturbation theory cutoff in a.u. (for TDA calculations) or in ${\rm a.u.}^2$ (for
    full TD-DFTB calculations) for the basis functions spanning the excited stated to be
    included. Every basis function whose cumulative interaction with basis
    functions included with the \texttt{-{}-energy\_threshold} criterion is
    larger than this threshold is included to the pruned basis.
\item \texttt{-{}-max\_memory}, \texttt{-g}: specifies the maximum memory the
    excited-state calculation can take. This is estimated from the size of the
    excited-state basis dimension and the number of desired roots. 
\item \texttt{-{}-scf\_mixer\_steering}: With this option, the
  method used to accelerate the convergence of the self-
consistent-field (SCF) calculations after an orbital steering mix can be set. Possible options are `no\_mixer` (no convergence acceleration), `diis`
(direct inversion of the iterative subspace, DIIS), `ediis` (energy DIIS), and `ediis\_diis`. The
default is `diis`.
\item \texttt{-{}-max\_iterations\_in\_steering}: This is used to
  specify the maximum number of SCF iterations after an orbital steering mix. Default is 100.
\end{itemize}


\section{Running \textsc{Sparrow} in Parallel}

By default, \textsc{Sparrow} will be compiled with \textsc{OpenMP} support and hence, it can be run in parallel. In order
to use multiple CPU cores, simply specify
\begin{verbatim}
export OMP_NUM_THREADS=n
\end{verbatim}
where \texttt{n} is the number of CPUs you want to use. Note that by default, \textsc{Sparrow} uses all available cores,
\textit{i.e.}, it will also run in parallel if you do not specify the above environment variable.



\chapter{Using the Python Bindings}

\textsc{Sparrow} provides Python bindings such that all functionality of \textsc{Sparrow} can be accessed also via the
Python programming language. In order to build the Python bindings, you need to specify \texttt{-DSCINE\_BUILD\_PYTHON\_BINDINGS=ON}
when running cmake (see also chapter~\nameref{ch:installation}). 
Some options, for instance, the convergence of the excited-state subspace solver, 
are only present in the Python bindings.
Please note that the setting types are rigorously checked in Python. It is
important not to confuse integer and floating point types!
For instance, there is a difference between 1 (an integer type) and 1.0 (a floating
point type).

In order to use the Python bindings, you need to specify the path to the Python library in the environment variable
\texttt{PYTHONPATH}, \textit{e.g.}, you have to run the command
\begin{verbatim}
export PYTHONPATH=$PYTHONPATH:<source code directory>/install/lib/python<version>/site-packages
\end{verbatim}
where \texttt{<version>} is the Python version you are using (\textit{e.g.}, 3.6). Now, you can simply import the library 
and use it as any other Python library. For example, in order to calculate the total electronic energy of H$_2$ with the 
\texttt{AM1} method as well as the gradient and the atomic second derivatives (where the latter can be used to approximate the Hessian as a block-diagonal matrix of second derivatives), use the following Python statements:
\begin{verbatim}
import scine_utilities as su
import scine_sparrow

# Generate Structure
structure = su.AtomCollection()
structure.elements = [su.ElementType.H, su.ElementType.H]
structure.positions = [[-0.7, 0, 0], [0.7, 0, 0]]

# Get calculator
manager = su.core.ModuleManager()
calculator = manager.get('calculator', 'AM1')

# Configure Calculator
calculator.structure = structure
calculator.set_required_properties([su.Property.Energy,
                                    su.Property.Gradients, 
                                    su.Property.AtomicHessians])

# Calculate
results = calculator.calculate()
print(results.energy)
print(results.gradients)
# Print the atomic second derivatives as a list of 3x3 matrices.
# To use this as an approximation of the total Hessian, this output
# needs to be rearranged in a block-diagonal matrix.
print(results.atomic_hessian.get_atomic_hessians())
\end{verbatim}
The output of \textsc{Sparrow} is always in Hartree atomic units, also the input is expected in Hartree atomic units.

A detailed list of all the functions provided by the \textsc{SCINE} Python libraries can be found by running
\begin{verbatim}
import scine_utilities

help(scine_utilities)
\end{verbatim}
The import of the SCINE \textsc{Sparrow} module mainly makes the included methods available.
A shorter version of the above code could be written using an XYZ file input.
\begin{verbatim}
import scine_utilities as su
import scine_sparrow

# Load xyz into calculator
manager = su.core.ModuleManager()
calculator = manager.get('calculator', 'AM1')
calculator.structure = su.io.read('h2.xyz')[0]

# Configure Calculator
calculator.set_required_properties([su.Property.Energy,su.Property.Gradients])

# calculate
results = calculator.calculate()
print(results.energy)
print(results.gradients)
\end{verbatim}
In the latter example, the XYZ file is expected in \AA{}, all results will still be in Hartree atomic units.

Excited-state calculations can also be calculated with the SCINE \textsc{Sparrow} module:
\begin{verbatim}
import scine_utilities as su
import scine_sparrow

# Generate structure
# This can also be done with 
# structure = su.io.read('structure.xyz')
# (will automatically be converted to bohr)
structure = su.AtomCollection(6)
structure.elements = [su.ElementType.C, su.ElementType.C, su.ElementType.H, 
                      su.ElementType.H, su.ElementType.H, su.ElementType.H]
# in angstrom                                                                
structure.positions = [[0.94815,    0.05810,    0.05008], 
                       [2.28393,    0.05810,    0.05008],       
                       [0.38826,   -0.56287,    0.74233],       
                       [0.38826,    0.67907,   -0.64217],       
                       [2.84382,   -0.56287,    0.74233],       
                       [2.84382,    0.67907,   -0.64217]]
structure.positions = structure.positions * su.BOHR_PER_ANGSTROM

# Get excited-state calculator
manager = su.core.ModuleManager()
excited_states_calculator = manager.get('calculator_with_reference', 'TD-DFTB')
# Generate a suitable ground-state reference calculator
excited_states_calculator.reference_calculator = manager.get('calculator', 'DFTB2')

# Configure the calculation
excited_states_calculator.reference_calculator.settings['self_consistence_criterion']=1e-9
excited_states_calculator.settings['number_eigenstates'] = 3
excited_states_calculator.settings['initial_subspace_dimension'] = 3

# Carry out the ground-state calculation
excited_states_calculator.reference_calculator.structure = structure
excited_states_calculator.reference_calculation()

# Carry out the excited-state calculation
result = excited_states_calculator.calculate()

# Print the vertical transition energies and the oscillator strengths
print(result.excited_states.singlet.eigenstates.eigenvalues * su.EV_PER_HARTREE)
su.transition_dipole_to_oscillator_strength(result.excited_states.singlet.transition_dipoles, 
                                            result.excited_states.singlet.eigenstates.eigenvalues)

# Print the labels of the transition density matrix elements
# and their coefficients in the first excited state
print(result.excited_states.mo_labels)
print(result.excited_states.singlet.eigenstates.eigenvectors[:,0])
\end{verbatim}

Orbital steering calculations can also be calculated with the SCINE \textsc{Sparrow} module:
\begin{verbatim}
import scine_utilities as su
import scine_sparrow

# Generate structure
# This can also be done with 
# structure = su.io.read('structure.xyz')
# (will automatically be converted to bohr)
structure = su.AtomCollection(6)
structure.elements = [su.ElementType.C, su.ElementType.C, su.ElementType.H, 
                      su.ElementType.H, su.ElementType.H, su.ElementType.H]
# in angstrom                                                                
structure.positions = [[0.94815,    0.05810,    0.05008], 
                       [2.28393,    0.05810,    0.05008],       
                       [0.38826,   -0.56287,    0.74233],       
                       [0.38826,    0.67907,   -0.64217],       
                       [2.84382,   -0.56287,    0.74233],       
                       [2.84382,    0.67907,   -0.64217]]
structure.positions = structure.positions * su.BOHR_PER_ANGSTROM

# Get orbital steering calculator
manager = su.core.ModuleManager()
orbital_steering_calculator = manager.get('calculator_with_reference',
'orbital_steering')
# Generate a suitable reference calculator
calc = manager.get('calculator', 'PM6')
calc.structure = structure
calc.settings['self_consistence_criterion'] = 1e-8
calc.settings['spin_mode'] = 'unrestricted'
orbital_steering_calculator.reference_calculator = calc

# Configure the calculation

# How often a steering of the orbitals is carried out (1 = every call to
"calculate()", 2 = every second call to "calculate()", ...)
# Often  steering the orbitals after each single point is not needed, and
# the additional cost is not justified. In those cases, one can carry out
# a steering calculation, e.g., every 5th single-point calculations. Since the
# density guess is taken to be the solution of the previous calculation, 
# a steered guess will provide a suitable solution for subsequent calculations.
orbital_steering_calculator.settings['mixing_frequency'] = 5
# How many occupied-virtual orbital pairs are created to mix
orbital_steering_calculator.settings['number_orbitals_to_mix'] = 15
# Note: not 0 and 90, as those would be integers.
orbital_steering_calculator.settings['minimal_mixing_angle'] = 0.0
orbital_steering_calculator.settings['maximal_mixing_angle'] = 90.0
# You can set other convergence criteria in the orbital steering calculations.
orbital_steering_calculator.settings['max_scf_iterations'] = 200 
orbital_steering_calculator.settings['scf_mixer'] = 'ediis_diis'

# Carry out ground-state calculation without steering (not needed, but you can
# do it). This will not increment the counter used to determine when a steering is
# needed.
orbital_steering_calculator.reference_calculation()

# Carry out the orbital-steering calculation
result = orbital_steering_calculator.calculate()

\end{verbatim}

\chapter{Extensions Planned in Future Releases}
\begin{itemize}
\item Availability of OMx models
\item Availability of the CISE approach
\item Implementation of multireference semiempirical approach
\item Implementation of periodic boundary conditions
\end{itemize}



\chapter{References}

Please consult the following references for more details on \textsc{Sparrow}.
We kindly ask you to cite the appropriate references in any publication of results obtained with \textsc{Sparrow}.
\vspace{1.0cm}

\begin{itemize}
\item Primary reference for Sparrow 3.1.0:
F.~Bosia, T.~Husch, C.~H.~M\"uller, S.~Polonius, J.-G.~Sobez, M.~Steiner, J.~P.~Unsleber, A.~C.~Vaucher, T.~Weymuth, M.~Reiher, \href{https://doi.org/10.5281/zenodo.3244105}{"qcscine/sparrow: Release 3.1.0 (Version 3.1.0)"}, Zenodo, 2022.
\item Presentation of the formalism of MNDO-type and OMx models: \newline
T.~Husch, A.~C.~Vaucher, M.~Reiher \href{https://doi.org/10.1002/qua.25799}{"Semiempirical Molecular Orbital Models Based on the Neglect of Diatomic Differential Overlap Approximation"}, \textit{Int.~J.~Quantum Chem.}, \textbf{2018}, \textit{118}, e25799.
\item Presentation of DFTB approaches: \newline
M.~Elstner, G.~Seifert, \href{https://doi.org/10.1098/rsta.2012.0483}{"Density functional tight binding"}, \textit{Phil. Trans.~R.~Soc.~A}, \textbf{2014}, \textit{371}, 20120483.
\item Presentation of CISE: \newline
T.~Husch, M.~Reiher \href{https://doi.org/10.1021/acs.jctc.8b00601}{"Comprehensive Analysis of the Neglect of Diatomic Differential Overlap Approximation"}, \textit{J.~Chem.~Theory Comput.}, \textbf{2018}, \textit{14}, 5169.
\item Presentation of the Orbital Steering approach: \newline
A.~C.~Vaucher, M.~Reiher
    \href{https://doi.org/10.1021/acs.jctc.7b00011}{"Steering Orbital
    Optimization out of Local Minima and Saddle Points Toward Lower Energy"},
    \textit{J.~Chem.~Theory Comput.}, \textbf{2017}, \textit{13}, 1219.
\end{itemize}

If you publish DFTB results, we kindly ask you to cite the references for the underlying parameters, which are specified in the README file in the respective parameter directory. The default parameters are mio-1-1 for DFTB2 and 3ob-3-1 for all other DFTB variations.


%%
% The back matter contains appendices, bibliographies, indices, glossaries, etc.

%\backmatter

%\printindex

\end{document}
